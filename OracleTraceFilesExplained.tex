%%%%%%%%%%%%%%%%%%%%%%%%%%%%%%%%%%%%%%%%%
% The Legrand Orange Book
% LaTeX Template
% Version 2.0 (9/2/15)
%
% This template has been downloaded from:
% http://www.LaTeXTemplates.com
%
% Mathias Legrand (legrand.mathias@gmail.com) with modifications by:
% Vel (vel@latextemplates.com)
%
% License:
% CC BY-NC-SA 3.0 (http://creativecommons.org/licenses/by-nc-sa/3.0/)
%
% Compiling this template:
% This template uses biber for its bibliography and makeindex for its index.
% When you first open the template, compile it from the command line with the 
% commands below to make sure your LaTeX distribution is configured correctly:
%
% 1) pdflatex main
% 2) makeindex main.idx -s StyleInd.ist
% 3) biber main
% 4) pdflatex main x 2
%
% After this, when you wish to update the bibliography/index use the appropriate
% command above and make sure to compile with pdflatex several times 
% afterwards to propagate your changes to the document.
%
% This template also uses a number of packages which may need to be
% updated to the newest versions for the template to compile. It is strongly
% recommended you update your LaTeX distribution if you have any
% compilation errors.
%
% Important note:
% Chapter heading images should have a 2:1 width:height ratio,
% e.g. 920px width and 460px height.
%
%%%%%%%%%%%%%%%%%%%%%%%%%%%%%%%%%%%%%%%%%

%----------------------------------------------------------------------------------------
%	PACKAGES AND OTHER DOCUMENT CONFIGURATIONS
%----------------------------------------------------------------------------------------

\documentclass[11pt,fleqn]{book} % Default font size and left-justified equations

%----------------------------------------------------------------------------------------
\input{structure} % Insert the structure.tex file which contains the majority of the structure behind the template

%----------------------------------------------------------------------------------------
%	NDunbar: INDEXING and Other Macros
%----------------------------------------------------------------------------------------
\input{macros.tex}

%----------------------------------------------------------------------------------------
% NDunbar: Put your PDF details here ....
%----------------------------------------------------------------------------------------
\input{PDFConfig.tex}



%----------------------------------------------------------------------------------------
% Main document starts here.
%----------------------------------------------------------------------------------------
\begin{document}


%----------------------------------------------------------------------------------------
%	TITLE PAGE
%----------------------------------------------------------------------------------------
\frontmatter			% Must be presnt if using an index and want to have correct indexing
                        % especially if mainmatter is used.
                        
\input{titlePage.tex}

%----------------------------------------------------------------------------------------
%	COPYRIGHT PAGE
%----------------------------------------------------------------------------------------

\newpage
~\vfill
\thispagestyle{empty}

\noindent Copyright \copyright 2017,2018 Norman Dunbar\\ % Copyright notice

\noindent \textsc{Published by MeMyselfEye Publishing ;-)}\\ % Publisher

\noindent The latest version of the eBook can be downloaded from \href{https://github.com/NormanDunbar/OracleTraceFilesExplained/releases/latest}{the book's github repository}.\\ % URL

\noindent Licensed under the Creative Commons Attribution-NonCommercial 3.0 Unported License (the ``License''). You may not use this file except in compliance with the License. You may obtain a copy of the License at \url{http://creativecommons.org/licenses/by-nc/3.0}. Unless required by applicable law or agreed to in writing, software distributed under the License is distributed on an \textsc{``as is'' basis, without warranties or conditions of any kind}, either express or implied. See the License for the specific language governing permissions and limitations under the License.\\ % License information

\noindent \textit{First printing, July 2017} \\% Printing/edition date

%\noindent This pdf document was created on \textit{\pdfcreationdate}.

\noindent This pdf document was created on \textit{\the\day/\the\month/\the\year} at \textit{\DTMcurrenttime}.

%----------------------------------------------------------------------------------------
%	TABLE OF CONTENTS
%----------------------------------------------------------------------------------------

\chapterimage{ChapterImage} % Table of contents heading image

\pagestyle{empty} % No headers


\tableofcontents % Print the table of contents itself
\listoftables
%\listoffigures
\lstlistoflistings

\cleardoublepage % Forces the first chapter to start on an odd page so it's on the right

\pagestyle{fancy} % Print headers again

%----------------------------------------------------------------------------------------
%Here come the contents of the book...
%----------------------------------------------------------------------------------------
\mainmatter
\input{includeContents}
\backmatter

%----------------------------------------------------------------------------------------
%	NDunbar: BIBLIOGRAPHY - NOT REQUIRED
%----------------------------------------------------------------------------------------

%\chapter*{Bibliography}
%\addcontentsline{toc}{chapter}{\textcolor{ocre}{Bibliography}}
%\section*{Books}
%\addcontentsline{toc}{section}{Books}
%\printbibliography[heading=bibempty,type=book]
%\section*{Articles}
%\addcontentsline{toc}{section}{Articles}
%\printbibliography[heading=bibempty,type=article]

%----------------------------------------------------------------------------------------
%	INDEX
%
% To make an index in MikeTex:
%
% Compile main file with pdfLaTeX.
% Compile main file with makeIndex -s StyleInd.ist.
% Compile main file with pdfLaTeX again.
%
%----------------------------------------------------------------------------------------

\cleardoublepage
\phantomsection
\setlength{\columnsep}{0.75cm}
\addcontentsline{toc}{chapter}{\textcolor{ocre}{Index}}
\printindex

%----------------------------------------------------------------------------------------

\end{document}
