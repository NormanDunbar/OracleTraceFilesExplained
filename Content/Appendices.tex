\begin{appendix}

%------------------------------------------------------------------------------
\chapter{Oracle Data Types}\label{oracle-data-types}
%------------------------------------------------------------------------------
\index{Oracle Data Types}

\section*{Bind Data Types}\label{bind-data-types}

The \inline{oacdty} parameter in a bind variables details determines
the data type of that bind variable. This is not necessarily the data
type of the column in a table that it may be being \inline{INSERT}ed
or \inline{UPDATE}d into, or compared against in a \inline{WHERE} clause.

There is a table of data types and codes in the \emph{Internal Data Types} section of the \emph{Data Types} chapter in the 12cR2 \href{http://docs.oracle.com/database/122/LNOCI/data-types.htm\#LNOCI16266}{\emph{Oracle Call Interface} manual}.

However, various other sources on the internet, and in books, seem to
disagree with some of what the above table shows. In addition, I have
come across many Oracle Trace files where a ROWID was code 11 and not
code 69. Consistency? Who mentioned consistency?

In 11g (possibly 10g) onwards, there is a new view named \inline{V\$SQL\_BIND\_CAPTURE} which lists various details about binds variables for SQL cursors. Two columns are of interest here, \inline{DATATYPE} and \inline{DATATYPE\_STRING} which we can use to convert a data type number into an actual data type name. In addition, we can look at the source code for the view named \inline{DBA\_TAB\_COLS}, or renamed to \inline{DBA\_TAB\_COLS\_V\$} in 12c.

\begin{lstlisting}[numbers=none,caption={Extracting the source code for DBA\_TAB\_COLS\_V\$ in Oracle 12.2.0.2}]
select text_vc
from   dba_views
where  view_name = 'DBA_TAB_COLS_V$';
\end{lstlisting}  

On older versions of Oracle, this should suffice:

\begin{lstlisting}[numbers=none,caption={Extracting the source code for DBA\_TAB\_COLS in Oracle 11.2.0.4}]
select text_vc
from   dba_views
where  view_name = 'DBA_TAB_COLS';
\end{lstlisting}  

You will only get the first 4000 characters, but this is enough to see the main DECODE statements which converts a data type to a data type name.

\begin{longtable}[]{@{}r|l@{}}
\toprule
Code & Data Type  \\
\midrule
\endhead
\bottomrule
\caption{Bind Variable data Types\ldots{}\textit{continues on next page}}
\endfoot
\caption{Bind Variable data Types}
\endlastfoot

  0 & UNDEFINED \\
  1 & NVARCHAR2 or VARCHAR2 \\
  2 & NUMBER or FLOAT(precision) \\
  8 & LONG \\
  9 & NCHAR VARYING or VARCHAR \\
 11 & ROWID \\
 12 & DATE \\
 16 & DATE \\
 23 & RAW \\
 24 & LONG RAW \\
 25 & UB2 \\
 29 & B4 \\
 58 & ANYDATA, XMLTYPE or OPAQUE \\
 69 & ROWID \\
 96 & NCHAR or CHAR \\
100 & BINARY\_FLOAT \\
101 & BINARY\_DOUBLE \\
102 & REF CURSOR \\
105 & MLSLABEL \\
106 & MLSLABEL \\
108 & User defined Types \\
111 & REF XMLTYPE \\
112 & NCLOB or CLOB \\
113 & BLOB \\
114 & BFILE \\
115 & CFILE \\
121 & User defined and/or object TYPEs \\
122 & User defined and/or object TYPEs, NESTED TABLE \\
123 & User defined and/or object TYPEs, VARRAY \\
178 & TIME(scale) \\
179 & TIME(scale) WITH TIME ZONE \\
180 & TIMESTAMP(scale) \\
181 & TIMESTAMP(scale) WITH TIME ZONE \\
182 & INTERVAL YEAR(precision) TO MONTH \\
183 & INTERVAL DAY(precision) TO SECOND(scale) \\
208 & UROWID (Universal ROWID)\\
231 & TIMESTAMP(scale) WITH LOCAL TIME ZONE \\

\bottomrule
\end{longtable}

Data type codes in the above table have been extracted from the \href{http://docs.oracle.com/database/122/LNOCI/data-types.htm\#LNOCI16266}{\emph{Oracle Call
Interface} manual}, numerous trace files and/or the views mentioned. There are, as noted, inconsistencies between the various sources which is why there are the occasional duplicate entry.Sadly, Oracle doesn't document some internal details as much as it should (or as much as \textit{I} think it should!).

\begin{note}
Data types 178 and 179 are both listed in 11g and 12c, but you try creating a table with the \inline{TIME(n)} or \inline{TIME(n) WITH TIME ZONE} and see what happens! They \emph{appear} to be Java data types as that appears to be the only manual that lists them as valid data types.
\end{note}

For an example of inconsistencies, here is a \inline{ROWID} with \inline{oacdty=11} as opposed to \inline{oacdty=69}. The following example was taken from a trace file, created on Windows with Oracle 11.2.0.4:

\begin{lstlisting}[numbers=none,caption={Bind Example - \inline{ROWID} With \inline{oacdty=11}}]
PARSING IN CURSOR #5141169152 len=37 dep=1 uid=0 oct=3 lid=0 tim=3520788504344 hv=1398610540 ad='7ffc0a95d898' sqlid='grwydz59pu6mc'
select text from view$ where rowid=:1
END OF STMT
PARSE #5141169152:c=0,e=14640,p=0,cr=29,cu=0,mis=1,r=0,dep=1,og=4,plh=0,tim=3520788504343
BINDS #5141169152:
 Bind#0
  oacdty=11 mxl=16(16) mxlc=00 mal=00 scl=00 pre=00
  oacflg=18 fl2=0001 frm=00 csi=00 siz=16 off=0
  kxsbbbfp=bff30890  bln=16  avl=16  flg=05
  value=00002787.000A.0001
\end{lstlisting}  
  
There's also an example of a \inline{REF\_CURSOR} bind on page \pageref{ref-cursor-102} of this eBook showing the use of an \inline{oacdty=102}.

%------------------------------------------------------------------------------
\chapter{Oracle Command Codes}\label{oracle-command-codes}
%------------------------------------------------------------------------------
\index{Oracle Command Codes}

\section*{Command Codes}\label{command-codes}

The \inline{oct} parameter in a \inline{PARSING IN CURSOR} line in an Oracle
trace file, determines the command that is being parsed in the SQL
statement. Why we should need this, I have no idea, as the SQL text for the
command will - obviously - show what command is being parsed. However\ldots{}

The following (large) table outlines the various command codes and was extracted
from an Oracle 12.2.0.1 database.

\begin{longtable}[]{@{}rl|rl@{}}
\toprule
Code & Command & Code & Command  \\
\midrule
\endhead
\bottomrule
\caption{Oracle Command Codes\ldots{}\textit{continues on next page}}
\endfoot
\caption{Oracle Command Codes}
\endlastfoot

0   & UNKNOWN                      & 1   & CREATE TABLE                 \\
2   & INSERT                       & 3   & SELECT                       \\
4   & CREATE CLUSTER               & 5   & ALTER CLUSTER                \\
6   & UPDATE                       & 7   & DELETE                       \\
8   & DROP CLUSTER                 & 9   & CREATE INDEX                 \\
10  & DROP INDEX                   & 11  & ALTER INDEX                  \\
12  & DROP TABLE                   & 13  & CREATE SEQUENCE              \\
14  & ALTER SEQUENCE               & 15  & ALTER TABLE                  \\
16  & DROP SEQUENCE                & 17  & GRANT OBJECT                 \\
18  & REVOKE OBJECT                & 19  & CREATE SYNONYM               \\
20  & DROP SYNONYM                 & 21  & CREATE VIEW                  \\
22  & DROP VIEW                    & 23  & VALIDATE INDEX               \\
24  & CREATE PROCEDURE             & 25  & ALTER PROCEDURE              \\
26  & LOCK                         & 27  & NO-OP                        \\
28  & RENAME                       & 29  & COMMENT                      \\
30  & AUDIT OBJECT                 & 31  & NOAUDIT OBJECT               \\
32  & CREATE DATABASE LINK         & 33  & DROP DATABASE LINK           \\
34  & CREATE DATABASE              & 35  & ALTER DATABASE               \\
36  & CREATE ROLLBACK SEG          & 37  & ALTER ROLLBACK SEG           \\
38  & DROP ROLLBACK SEG            & 39  & CREATE TABLESPACE            \\
40  & ALTER TABLESPACE             & 41  & DROP TABLESPACE              \\
42  & ALTER SESSION                & 43  & ALTER USER                   \\
44  & COMMIT                       & 45  & ROLLBACK                     \\
46  & SAVEPOINT                    & 47  & PL/SQL EXECUTE               \\
48  & SET TRANSACTION              & 49  & ALTER SYSTEM                 \\
50  & EXPLAIN                      & 51  & CREATE USER                  \\
52  & CREATE ROLE                  & 53  & DROP USER                    \\
54  & DROP ROLE                    & 55  & SET ROLE                     \\
56  & CREATE SCHEMA                & 57  & CREATE CONTROL FILE          \\
59  & CREATE TRIGGER               & 60  & ALTER TRIGGER                \\
61  & DROP TRIGGER                 & 62  & ANALYZE TABLE                \\
63  & ANALYZE INDEX                & 64  & ANALYZE CLUSTER              \\
65  & CREATE PROFILE               & 66  & DROP PROFILE                 \\
67  & ALTER PROFILE                & 68  & DROP PROCEDURE               \\
70  & ALTER RESOURCE COST          & 71  & CREATE MATERIALIZED VIEW LOG \\
72  & ALTER MATERIALIZED VIEW LOG  & 73  & DROP MATERIALIZED VIEW LOG   \\
74  & CREATE MATERIALIZED VIEW     & 75  & ALTER MATERIALIZED VIEW      \\
76  & DROP MATERIALIZED VIEW       & 77  & CREATE TYPE                  \\
78  & DROP TYPE                    & 79  & ALTER ROLE                   \\
80  & ALTER TYPE                   & 81  & CREATE TYPE BODY             \\
82  & ALTER TYPE BODY              & 83  & DROP TYPE BODY               \\
84  & DROP LIBRARY                 & 85  & TRUNCATE TABLE               \\
86  & TRUNCATE CLUSTER             & 88  & ALTER VIEW                   \\
90  & SET CONSTRAINTS              & 91  & CREATE FUNCTION              \\
92  & ALTER FUNCTION               & 93  & DROP FUNCTION                \\
94  & CREATE PACKAGE               & 95  & ALTER PACKAGE                \\
96  & DROP PACKAGE                 & 97  & CREATE PACKAGE BODY          \\
98  & ALTER PACKAGE BODY           & 99  & DROP PACKAGE BODY            \\
100 & LOGON                        & 101 & LOGOFF                       \\
102 & LOGOFF BY CLEANUP            & 103 & SESSION REC                  \\
104 & SYSTEM AUDIT                 & 105 & SYSTEM NOAUDIT               \\
106 & AUDIT DEFAULT                & 107 & NOAUDIT DEFAULT              \\
108 & SYSTEM GRANT                 & 109 & SYSTEM REVOKE                \\
110 & CREATE PUBLIC SYNONYM        & 111 & DROP PUBLIC SYNONYM          \\
112 & CREATE PUBLIC DATABASE LINK  & 113 & DROP PUBLIC DATABASE LINK    \\
114 & GRANT ROLE                   & 115 & REVOKE ROLE                  \\
116 & EXECUTE PROCEDURE            & 117 & USER COMMENT                 \\
118 & ENABLE TRIGGER               & 119 & DISABLE TRIGGER              \\
120 & ENABLE ALL TRIGGERS          & 121 & DISABLE ALL TRIGGERS         \\
122 & NETWORK ERROR                & 123 & EXECUTE TYPE                 \\
125 & READ DIRECTORY               & 126 & WRITE DIRECTORY              \\
128 & FLASHBACK                    & 129 & CREATE SESSION               \\
130 & ALTER MINING MODEL           & 131 & SELECT MINING MODEL          \\
133 & CREATE MINING MODEL          & 134 & ALTER PUBLIC SYNONYM         \\
135 & DIRECTORY EXECUTE            & 136 & SQL*LOADER DIRECT PATH LOAD  \\
137 & DATAPUMP DIRECT PATH UNLOAD  & 138 & DATABASE STARTUP             \\
139 & DATABASE SHUTDOWN            & 140 & CREATE SQL TXLN PROFILE      \\
141 & ALTER SQL TXLN PROFILE       & 142 & USE SQL TXLN PROFILE         \\
143 & DROP SQL TXLN PROFILE        & 144 & CREATE MEASURE FOLDER        \\
145 & ALTER MEASURE FOLDER         & 146 & DROP MEASURE FOLDER          \\
147 & CREATE CUBE BUILD PROCESS    & 148 & ALTER CUBE BUILD PROCESS     \\
149 & DROP CUBE BUILD PROCESS      & 150 & CREATE CUBE                  \\
151 & ALTER CUBE                   & 152 & DROP CUBE                    \\
153 & CREATE CUBE DIMENSION        & 154 & ALTER CUBE DIMENSION         \\
155 & DROP CUBE DIMENSION          & 157 & CREATE DIRECTORY             \\
158 & DROP DIRECTORY               & 159 & CREATE LIBRARY               \\
160 & CREATE JAVA                  & 161 & ALTER JAVA                   \\
162 & DROP JAVA                    & 163 & CREATE OPERATOR              \\
164 & CREATE INDEXTYPE             & 165 & DROP INDEXTYPE               \\
166 & ALTER INDEXTYPE              & 167 & DROP OPERATOR                \\
168 & ASSOCIATE STATISTICS         & 169 & DISASSOCIATE STATISTICS      \\
170 & CALL METHOD                  & 171 & CREATE SUMMARY               \\
172 & ALTER SUMMARY                & 173 & DROP SUMMARY                 \\
174 & CREATE DIMENSION             & 175 & ALTER DIMENSION              \\
176 & DROP DIMENSION               & 177 & CREATE CONTEXT               \\
178 & DROP CONTEXT                 & 179 & ALTER OUTLINE                \\
180 & CREATE OUTLINE               & 181 & DROP OUTLINE                 \\
182 & UPDATE INDEXES               & 183 & ALTER OPERATOR               \\
187 & CREATE SPFILE                & 188 & CREATE PFILE                 \\
189 & MERGE                        & 190 & PASSWORD CHANGE              \\
192 & ALTER SYNONYM                & 193 & ALTER DISKGROUP              \\
194 & CREATE DISKGROUP             & 195 & DROP DISKGROUP               \\
197 & PURGE USER\_RECYCLEBIN       & 198 & PURGE DBA\_RECYCLEBIN        \\
199 & PURGE TABLESPACE             & 200 & PURGE TABLE                  \\
201 & PURGE INDEX                  & 202 & UNDROP OBJECT                \\
202 & UNDROP OBJECT                & 203 & DROP DATABASE                \\
204 & FLASHBACK DATABASE           & 205 & FLASHBACK TABLE              \\
206 & CREATE RESTORE POINT         & 207 & DROP RESTORE POINT           \\
208 & PROXY AUTHENTICATION ONLY    & 209 & DECLARE REWRITE EQUIVALENCE  \\
210 & ALTER REWRITE EQUIVALENCE    & 211 & DROP REWRITE EQUIVALENCE     \\
212 & CREATE EDITION               & 213 & ALTER EDITION                \\
214 & DROP EDITION                 & 215 & DROP ASSEMBLY                \\
216 & CREATE ASSEMBLY              & 217 & ALTER ASSEMBLY               \\
218 & CREATE FLASHBACK ARCHIVE     & 219 & ALTER FLASHBACK ARCHIVE      \\
220 & DROP FLASHBACK ARCHIVE       & 221 & DEBUG CONNECT                \\
223 & DEBUG PROCEDURE              & 225 & ALTER DATABASE LINK          \\
226 & CREATE PLUGGABLE DATABASE    & 227 & ALTER PLUGGABLE DATABASE     \\
228 & DROP PLUGGABLE DATABASE      & 229 & CREATE AUDIT POLICY          \\
230 & ALTER AUDIT POLICY           & 231 & DROP AUDIT POLICY            \\
232 & CODE-BASED GRANT             & 233 & CODE-BASED REVOKE            \\
234 & CREATE LOCKDOWN PROFILE      & 235 & DROP LOCKDOWN PROFILE        \\
236 & ALTER LOCKDOWN PROFILE       & 237 & TRANSLATE SQL                \\
238 & ADMINISTER KEY MANAGEMENT    & 239 & CREATE MATERIALIZED ZONEMAP  \\
240 & ALTER MATERIALIZED ZONEMAP   & 241 & DROP MATERIALIZED ZONEMAP    \\
242 & DROP MINING MODEL            & 243 & CREATE ATTRIBUTE DIMENSION   \\
244 & ALTER ATTRIBUTE DIMENSION    & 245 & DROP ATTRIBUTE DIMENSION     \\
246 & CREATE HIERARCHY             & 247 & ALTER HIERARCHY              \\
248 & DROP HIERARCHY               & 249 & CREATE ANALYTIC VIEW         \\
250 & ALTER ANALYTIC VIEW          & 251 & DROP ANALYTIC VIEW           \\
305 & ALTER PUBLIC DATABASE LINK   \\

\bottomrule
\end{longtable}

The exact list of commands for your particular database version can be
extracted using the following SQL command:

\begin{lstlisting}[language=SQL,caption={SQL Query to List Oracle Command Codes}]
select action as code,
       name as command
from   audit_actions;
\end{lstlisting}

There are 239 different command codes in Oracle 12.2.0.1, 212 in Oracle 12.1.0.2 while Oracle 11g (11.2.0.4) has only (!) 181.

%------------------------------------------------------------------------------
\chapter{Oracle Characterset Codes}\label{oracle-characterset-codes}
%------------------------------------------------------------------------------
\index{Oracle Character Set Codes}

\section*{Bind Charactersets}\label{bind-charactersets}

Some data types use different character sets. These are coded in the \inline{csi} field in the bind details lines of the trace file. You can extract the list of current character set codes and names with the following query:

\begin{lstlisting}[language=SQL,caption={SQL Query to list Character Set Codes and Names}]
select nls_charset_id(value), value 
from   v$nls_valid_values
where  isdeprecated='FALSE'
and    parameter = 'CHARACTERSET'
order  by nls_charset_id(value);
\end{lstlisting}

The values that you may see here are as follows, taken from an Oracle 12.2.0.1 database where there are 222 character sets listed, the same number as in Oracle 12.1.0.1.


\begin{longtable}[]{@{}rl|rl|rl@{}}
\toprule
Code & Character Set & Code & Character Set & Code & Character Set \\
\midrule
\endhead
\bottomrule
\caption{Oracle Character Set Codes\ldots{}\textit{continues on next page}}
\endfoot
\caption{Oracle Character Set Codes}
\endlastfoot

1   & US7ASCII       & 159 & CL8MACCYRILLICS   & 301  & EE8EBCDIC870C    \\
2   & WE8DEC         & 160 & WE8PC860          & 312  & TR8EBCDIC1026S   \\
3   & WE8HP          & 161 & IS8PC861          & 314  & BLT8EBCDIC1112S  \\
4   & US8PC437       & 162 & EE8MACCES         & 315  & IW8EBCDIC424S    \\
5   & WE8EBCDIC37    & 163 & EE8MACCROATIANS   & 316  & EE8EBCDIC870S    \\
6   & WE8EBCDIC500   & 164 & TR8MACTURKISHS    & 317  & CL8EBCDIC1025S   \\
7   & WE8EBCDIC1140  & 165 & IS8MACICELANDICS  & 319  & TH8TISEBCDICS    \\
8   & WE8EBCDIC285   & 166 & EL8MACGREEKS      & 320  & AR8EBCDIC420S    \\
9   & WE8EBCDIC1146  & 167 & IW8MACHEBREWS     & 322  & CL8EBCDIC1025C   \\
10  & WE8PC850       & 170 & EE8MSWIN1250      & 323  & CL8EBCDIC1025R   \\
11  & D7DEC          & 171 & CL8MSWIN1251      & 324  & EL8EBCDIC875R    \\
12  & F7DEC          & 172 & ET8MSWIN923       & 325  & CL8EBCDIC1158    \\
13  & S7DEC          & 173 & BG8MSWIN          & 326  & CL8EBCDIC1158R   \\
14  & E7DEC          & 174 & EL8MSWIN1253      & 327  & EL8EBCDIC423R    \\
15  & SF7ASCII       & 175 & IW8MSWIN1255      & 351  & WE8MACROMAN8     \\
16  & NDK7DEC        & 176 & LT8MSWIN921       & 352  & WE8MACROMAN8S    \\
17  & I7DEC          & 177 & TR8MSWIN1254      & 353  & TH8MACTHAI       \\
18  & NL7DEC         & 178 & WE8MSWIN1252      & 354  & TH8MACTHAIS      \\
19  & CH7DEC         & 179 & BLT8MSWIN1257     & 368  & HU8CWI2          \\
20  & YUG7ASCII      & 180 & D8EBCDIC273       & 380  & EL8PC437S        \\
21  & SF7DEC         & 181 & I8EBCDIC280       & 381  & EL8EBCDIC875     \\
22  & TR7DEC         & 182 & DK8EBCDIC277      & 382  & EL8PC737         \\
23  & IW7IS960       & 183 & S8EBCDIC278       & 383  & LT8PC772         \\
25  & IN8ISCII       & 184 & EE8EBCDIC870      & 384  & LT8PC774         \\
27  & WE8EBCDIC1148  & 185 & CL8EBCDIC1025     & 385  & EL8PC869         \\
28  & WE8PC858       & 186 & F8EBCDIC297       & 386  & EL8PC851         \\
31  & WE8ISO8859P1   & 187 & IW8EBCDIC1086     & 390  & CDN8PC863        \\
32  & EE8ISO8859P2   & 188 & CL8EBCDIC1025X    & 401  & HU8ABMOD         \\
33  & SE8ISO8859P3   & 189 & D8EBCDIC1141      & 500  & AR8ASMO8X        \\
34  & NEE8ISO8859P4  & 190 & N8PC865           & 554  & AR8NAFITHA711    \\
35  & CL8ISO8859P5   & 191 & BLT8CP921         & 555  & AR8SAKHR707      \\
36  & AR8ISO8859P6   & 192 & LV8PC1117         & 556  & AR8MUSSAD768     \\
37  & EL8ISO8859P7   & 193 & LV8PC8LR          & 557  & AR8ADOS710       \\
38  & IW8ISO8859P8   & 194 & BLT8EBCDIC1112    & 558  & AR8ADOS720       \\
39  & WE8ISO8859P9   & 195 & LV8RST104090      & 559  & AR8APTEC715      \\
40  & NE8ISO8859P10  & 196 & CL8KOI8R          & 560  & AR8MSWIN1256     \\
41  & TH8TISASCII    & 197 & BLT8PC775         & 561  & AR8NAFITHA721    \\
42  & TH8TISEBCDIC   & 198 & DK8EBCDIC1142     & 563  & AR8SAKHR706      \\
43  & BN8BSCII       & 199 & S8EBCDIC1143      & 565  & AR8ARABICMAC     \\
44  & VN8VN3         & 200 & I8EBCDIC1144      & 566  & AR8ARABICMACS    \\
45  & VN8MSWIN1258   & 201 & F7SIEMENS9780X    & 590  & LA8ISO6937       \\
46  & WE8ISO8859P15  & 202 & E7SIEMENS9780X    & 829  & JA16VMS          \\
47  & BLT8ISO8859P13 & 203 & S7SIEMENS9780X    & 830  & JA16EUC          \\
48  & CEL8ISO8859P14 & 204 & DK7SIEMENS9780X   & 831  & JA16EUCYEN       \\
49  & CL8ISOIR111    & 205 & N7SIEMENS9780X    & 832  & JA16SJIS         \\
50  & WE8NEXTSTEP    & 206 & I7SIEMENS9780X    & 833  & JA16DBCS         \\
51  & CL8KOI8U       & 207 & D7SIEMENS9780X    & 834  & JA16SJISYEN      \\
52  & AZ8ISO8859P9E  & 208 & F8EBCDIC1147      & 835  & JA16EBCDIC930    \\
70  & AR8EBCDICX     & 210 & WE8GCOS7          & 836  & JA16MACSJIS      \\
81  & EL8DEC         & 211 & EL8GCOS7          & 837  & JA16EUCTILDE     \\
82  & TR8DEC         & 221 & US8BS2000         & 838  & JA16SJISTILDE    \\
90  & WE8EBCDIC37C   & 222 & D8BS2000          & 840  & KO16KSC5601      \\
91  & WE8EBCDIC500C  & 223 & F8BS2000          & 842  & KO16DBCS         \\
92  & IW8EBCDIC424   & 224 & E8BS2000          & 845  & KO16KSCCS        \\
93  & TR8EBCDIC1026  & 225 & DK8BS2000         & 846  & KO16MSWIN949     \\
94  & WE8EBCDIC871   & 226 & S8BS2000          & 850  & ZHS16CGB231280   \\
95  & WE8EBCDIC284   & 230 & WE8BS2000E        & 851  & ZHS16MACCGB23128 \\
96  & WE8EBCDIC1047  & 231 & WE8BS2000         & 852  & ZHS16GBK        \\
97  & WE8EBCDIC1140C & 232 & EE8BS2000         & 853  & ZHS16DBCS       \\
98  & WE8EBCDIC1145  & 233 & CE8BS2000         & 854  & ZHS32GB18030    \\
99  & WE8EBCDIC1148C & 235 & CL8BS2000         & 860  & ZHT32EUC        \\
100 & WE8EBCDIC1047E & 239 & WE8BS2000L5       & 861  & ZHT32SOPS       \\
101 & WE8EBCDIC924   & 241 & WE8DG             & 862  & ZHT16DBT        \\
110 & EEC8EUROASCI   & 251 & WE8NCR4970        & 863  & ZHT32TRIS       \\
113 & EEC8EUROPA3    & 261 & WE8ROMAN8         & 864  & ZHT16DBCS       \\
114 & LA8PASSPORT    & 262 & EE8MACCE          & 865  & ZHT16BIG5       \\
140 & BG8PC437S      & 263 & EE8MACCROATIAN    & 866  & ZHT16CCDC       \\
150 & EE8PC852       & 264 & TR8MACTURKISH     & 867  & ZHT16MSWIN950   \\
152 & RU8PC866       & 265 & IS8MACICELANDIC   & 868  & ZHT16HKSCS      \\
153 & RU8BESTA       & 266 & EL8MACGREEK       & 871  & UTF8            \\
154 & IW8PC1507      & 267 & IW8MACHEBREW      & 872  & UTFE            \\
155 & RU8PC855       & 277 & US8ICL            & 873  & AL32UTF8        \\
156 & TR8PC857       & 278 & WE8ICL            & 992  & ZHT16HKSCS31    \\
158 & CL8MACCYRILLIC & 279 & WE8ISOICLUK       & 2000 & AL16UTF16       \\

\bottomrule
\end{longtable}

If you see a character set code in the \inline{csi} field, and you don't have the above list, you can determine the characterset in use with the following query:

\begin{lstlisting}[language=SQL,caption={SQL Query to Convert a \inline{csi} Code to a Charcter Set Name}]
SELECT NLS_CHARSET_NAME(1) FROM dual;
\end{lstlisting}

Correspondingly, you can go from a charcter set name to its \inline{csi} code with the following query:

\begin{lstlisting}[language=SQL,caption={SQL Query to Convert a Charcter Set Name  to a \inline{csi} Code}]
SELECT NLS_CHARSET_ID('US7ASCII') FROM dual;
\end{lstlisting}

%------------------------------------------------------------------------------
\chapter{How this Book Evolved}
%------------------------------------------------------------------------------
\label{how-this-book-evolved}%

\section{Why this Book?}\label{why-this-book}

This eBook originally started as a collection of useful notes, scripts, etc which I had come across, written or deduced, over the years of playing with Oracle Trace Files and various Trace File Analysis tools.

Eventually, I collected them all together in one place so that I could have all the useful information, in one easy to find place\footnote{I'm getting old, every day it seems. I am starting to forget things I knew - they are ageing out of the cache! I now appear to have the memory capacity of a small newt, so I need to have things written down!}.

There are many places on the internet, people I have met or worked with, books which I have bought (or had bought for me) and read, to whom I will always be grateful as they have taught me many things. Maybe I can help others as I have been helped.

As Isaac Newton\index{Newton, Isaac}\person{Newton, Isaac} is thought to have said, ``\emph{I have stood on the shoulders of giants}'' - me too.


\section{Creating the Book}\label{creating-the-book}

The book was created in plain text files, originally in \href{https://en.wikipedia.org/wiki/ReStructuredText}{ReStructuredText} mode, written (and edited) on both Windows\footnote{At work, in my lunch hour}, or on Linux - which I use for all my personal and business needs at home.

These RST files were simply a quick way to gather notes. They were then later enhanced by adding more detail and/or example code, and converted to \LaTeX{}\program{\LaTeX{}} by judicious use of the \href{https://pandoc.org/}{Pandoc}\program{Pandoc} text conversion utility, which I \emph{highly} recommend.

The converted \LaTeX{}\program{\LaTeX{}} files were then further enhanced, indexed, tidied up, etc using \href{http://www.texstudio.org/}{TexStudio}\program{TexStudio} which runs on both Windows and Linux, so I had the same development environment in both locations. Handy. 

The book itself, was created using a \LaTeX{}\program{\LaTeX{}} template called [the] \href{https://www.latextemplates.com/template/the-legrand-orange-book}{Legrand Orange Book Template} created by Mathias Legrand\index{Legrand, Mathias}. It is thanks to him that you get this book for free, because the licence terms of the book template specify no commercial use. I'm happy with that myself.

The front cover image on this book is taken from the book \emph{Kunstformen der Natur}\index{Kunstformen der Natur} by German biologist Ernst Haeckel\index{Haeckel, Ernst}\person{Haeckel, Ernst}. The book was published between 1899 and 1904. The image used is of various \emph{Discomedusae}\index{Discomedusae} which are a taxonomic group of jellyfish.

You can read about them on \href{https://en.wikipedia.org/wiki/Discomedusae}{Wikipedia} and there is a brief overview of the above book, also on \href{https://en.wikipedia.org/wiki/Kunstformen_der_Natur}{Wikipedia},
which shows a number of other images taken from the book. (Some of which I considered before choosing the current one!)

\emph{Discomedusae}\index{Discomedusae}\footnote{DiscoMedusae? Sounds like dancing jellyfish to me!} have absolutely nothing to do with Oracle or Trace Files - but I liked the image, and decided that it would make a good cover for the book and a decent enough chapter heading image too.

%------------------------------------------------------------------------------
\chapter{TraceAdjust Utility}
%------------------------------------------------------------------------------
\label{traceadjust}\program{TraceAdjust}

\section*{TraceAdjust Introduction}\label{traceadjust-introduction}


\emph{TraceAdjust} is a utility that I wrote to help me process the
myriads of trace files that I come across in my DBA work. You can get
the source code from github in my 
\href{https://github.com/NormanDunbar/TraceAdjust}{TraceAdjust repository} and compile it
on Windows or Linux/Unix with any decent C++ compiler.

It reads a normal trace file and writes out an adjusted one, as follows:

\begin{itemize}
\tightlist
\item
  The \inline{tim} values are converted to seconds, by inserting a
  decimal point in the appropriate position;
\item
  It adds a \inline{delta} to each \inline{tim} line. The delta is
  the number of microseconds between the \inline{tim} on this line,
  and the \inline{tim} on the previous (appropriate) trace line. This
  allows me to see how long passed between the previous \inline{tim}
  and this one. Occasionally useful.
\item
  It adds a \inline{dslt} to each \inline{tim} line. This is the
  ``delta since last timestamp'' and simply counts up the number of
  microseconds that have passed since the trace file last produced a
  timestamp line similar to the one in the header. Again, occasionally
  useful.
\item
  It adds a \inline{local} to each \inline{tim} line. This is a
  conversion of the \inline{tim} value on the line, to an actual
  date, in the current local timezone. The time part is resolved down to
  microsecond level. This is usually very useful!
\end{itemize}

Running a trace file through TraceAdjust\program{TraceAdjust} will create a new trace file,
which some trace analysing utilities cannot cope with due to the
additional fields that I have introduced. The Trace File Browser, in
Toad\program{Toad}, on the other hand, copes with my trace files quite happily and
simply ignores the additional data as appropriate.

The example below shows the before and after state of a \inline{PARSE} line from a trace file:


\begin{lstlisting}[numbers=none,caption={TraceAdjust Example - Before Processing}]
PARSE #4474286416:c=0,e=418,p=0,cr=0,cu=0,mis=1,r=0,dep=1,og=4,plh=0,tim=1030574627220
\end{lstlisting}

Which becomes:

\begin{lstlisting}[numbers=none,caption={TraceAdjust Example - After Processing}]
PARSE #4474286416:c=0,e=418,p=0,cr=0,cu=0,mis=1,r=0,dep=1,og=4,plh=0,tim=1030574.627220,delta=-1,dslt=768371,local='2017 Mar 13 09:23:21.768371'
\end{lstlisting}

%------------------------------------------------------------------------------
\chapter{TraceMiner2 Utility}
%------------------------------------------------------------------------------
\label{traceminer2-utility}
\program{TraceMiner2}

\section*{TraceMiner2 Introduction}\label{traceminer2-introduction}

So, you have a trace file, chock full of statements with lots of bind
variables in use. You \emph{need} to read through it to find out which
execution of any of the statements used which actual bind values. How is
this easily done?

\emph{TraceMiner2}\program{TraceMiner2}\footnote{There will be no more unashamed plugs in this
  document, I promise.} is another utility that I wrote to help me
process the myriads of trace files that I come across in my DBA work.

You can get the source code from github in my 
\href{https://github.com/NormanDunbar/TraceMiner2}{TraceMiner2 repository} and compile it
on Windows or Linux/Unix with any decent C++ compiler.

It reads a trace file and writes out an HTML report showing various
details of the SQL in the file, showing:

\begin{itemize}
\tightlist
\item
  The line number of the file where the SQL statement was found at
  (\inline{PARSING IN CURSOR});
\item
  The line number of the file where  the \inline{PARSE} statement was found at;
\item
  The line number of the file where  the \inline{BINDS} details were found at;
\item
  The line number of the file where  the \inline{EXEC} statement was found at;
\item
  The depth (\inline{dep}) of the statement when parsed etc.
\end{itemize}

You can choose to ignore statements over any given depth, so if you only
want top-level and \inline{dep=1} statements, just request \inline{--depth=1} on the
command line. 

The default HTML report appears as follows:

\includegraphics[width=\textwidth]{Content/images/TraceMiner2.png}

However, you can, if you have a standard report format at your company,
configure the generated css file to match that of your format.
\emph{TraceMiner2}\program{TraceMiner2} will not overwrite the css file if one exists in the
output folder.



\end{appendix}
