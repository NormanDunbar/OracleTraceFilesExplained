\chapter{Introduction}\label{introduction}

Oracle trace files are not greatly documented. This document is an attempt to do so. It is \emph{not} official in any way and is based on a good few years of reading these files to help diagnose various database problems.

A trace file is really the best way to delve into what Oracle is doing, or to discover why something is taking so long - it shows you exactly what happened during the period that the session was being traced.

Even better, when you extract an Explain Plan from a trace file, it is showing you exactly how Oracle retrieved the data from the tables, and exactly where the time was spend in doing so. Running an \inline{Explain Plan for ...} statement in \emph{SQL*Plus}\program{SQL*Plus}, \emph{Toad}\program{Toad} etc tells you what Oracle \emph{might} do. The two are not always the same.

The trace files used (and abused) in this eBook were all self contained, and extracted from a dedicated session on the database. If you are using shared sessions, then you have the unfortunate problem of having bits of your session trace spread over any number of different trace files. Not fun.

From Oracle 10g onwards, a utility has been supplied, named \emph{trcsess}\program{Trcsess}. This allows you to collect together all the separate files and merge them into one, ready for analysis. Even better, it apparently can also merge Oracle 9i shared server session trace files. Bonus!

\section*{Trace File Extracts}

In the remainder of this eBook, extracts from trace files will be shown as follows:

\begin{lstlisting}[numbers=none]
WAIT #3220341128: nam='db file sequential read' ela= 1023 file#=3 block#=12 blocks=1 obj#=-1 tim=3520817183625
\end{lstlisting}

You will notice that long lines from the trace file are wrapped in the example above. However, all such wrapped lines are indicated by a \textbf{$\Longrightarrow$} at the start of the continuation line(s). Sometimes though, the break in the lines can be at an awkward position. Sadly, I can't help this, it's all automagically done and I have no input as to how, exactly, \LaTeX{} decides.

\section*{Code Listings}

SQL scripts (or any other program listings) will be listed as per the example below. The only difference being line numbers, in case I have to reference a particular point, and syntax highlighting.

\begin{lstlisting}[language=SQL]
select name, parameter1, parameter2, parameter3
from   v$event_name
where  name = 'db file sequential read';
\end{lstlisting}

\section*{Irregular Updates}

This eBook is a work in progress, I am always learning, or trying to. As I update it, the latest version will always appear on GitHub as detailed on the Copyright page above. Also on that page, you will be able to see the date and time that the version of the eBook you are reading, was generated. That's done automagically too, so I don't need to remember to update it - thankfully!

\section*{Comments}

Seen anything blatantly wrong? Something  you think needs a better explanation? Too many blatant plugs? You can either create an issue on GitHub, or, \href{mailto://norman@dunbar-it.co.uk}{email me} with details and I'll do my best to get it fixed.